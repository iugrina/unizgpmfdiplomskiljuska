%%%%%%%%%%%%%%%%%%%%%%%%%%%%%%%%%%%%%%%%%%%
% Ovdje napisite naziv drugog Sectiona
\Section{Drugi naslov} \label{Sec:2}
%
% Ispod napisite tekst Sectiona

% Na primjer
\begin{equation}
F(x,y,1)=f(x,y) \mathrm{.}
\end{equation}
Sada je o\v{c}ito da $F=0$ i $f=0$ sadr\v{z}e iste to\v{c}ke euklidske ravnine. Krivulju $F=0$
zovemo \bi{pro\v{s}irenje} krivulje $f$ u projektivnu ravninu ili jednostavnije pro\v{s}irenje
od $f$.

Broj presjeka izme\dj{}u dvije krivulje u ishodi\v{s}tu ne bi
se trebao promijeniti ako krivulju restriktiramo iz projektivne ravnine u
euklidsku ravninu i zamjenimo homogene koordinate sa uobi\v{c}ajnim $(x,y)$ koordinatama.

