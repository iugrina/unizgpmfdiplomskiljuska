%%%%%%%%%%%%%%%%%%%%%%%%%%%%%%%%%%%%%%%%%%%
%
\section{Rangovi u usporedbi dva tretmana} \label{sec:1p1}
%
\setcounter{equation}{0}

U razli\v{c}itim se podru\v{c}jima \v{c}esto pojavljuje 
problem je li predlo\v{z}ena inovacija bolja od trenutnog rje\v{s}enja.
"Produ\v{z}uje li novi lijek \v{z}ivot pacijenata s malignim tumorima?",
"Smanjuje li se \v{s}tetan u\v{c}inak cigareta uporabom odre\dj{}enog filtera?"
i "Smanjuje li novi sustav prijevoza tro\v{s}kove?"
samo su neki od primjera.
Kroz sljedeći primjer poku\v{s}ati \'{c}emo ilustrirati na\v{c}in na
koji, barem intuitivno, mo\v{z}emo testirati ovakva pitanja.

\primjer{
Bolnica za ljude s mentalnim pote\v{s}ko\'{c}ama \v{z}eli testirati novi lijek
koji navodno ima pozitivan utjecaj na neki mentalni poreme\'{c}aj. U bolnici
se trenutno nalazi 5 pacijenata s tim poreme\'{c}ajem u, otprilike, istom
stadiju. Od ovih pet pacijenata tri su odabrana na slu\v{c}ajan na\v{c}in
da prime novi lijek dok \'{c}e preostala dva slu\v{z}iti kao kontrolna grupa.
Oni \'{c}e primiti obi\v{c}nu tabletu bez ikakvih aktivnih sastojaka (takav
"lijek" zovemo \textit{placebo}). Pacijenti (po mogu\'{c}nosti
i osoblje) ne\'{c}e znati da li su dobili novi lijek. Ovo eliminira mogu\'{c}nost
psiholo\v{s}kog utjecaja.

Nakon nekog vremena nezavisni psihijatar pregledava pacijente te ih rangira
po stupnju njihovog stanja. Pacijent za \v{c}ije se stanje procijeni da je
najgore dobije rang 1, sljede\'{c}i 2, \dots, do ranga 5.

Pretpostavimo da lijek nema utjecaja, tj. da se pacijentovo stanje ne mijenja
bez obzira je li primio novi lijek ili placebo. Ovu tvrdnju nazovimo
\textit{nultom hipotezom}. Kako pod nultom hipotezom rang koji pacijenti dobivaju
ne ovisi o lijeku, ve\'{c} samo o stanju pacijenta, jasno je da izbor pacijenata
koji \'{c}e primiti novi lijek ne utje\v{c}e na rangiranje. Dakle, mo\v{z}emo promatrati
svaki od rangova kao odre\dj{}en i prije nego \v{s}to smo izabrali testnu i kontrolnu grupu.
Izbor pacijenata koji \'{c}e biti u testnoj grupi tada u potpunosti odre\dj{}uje i rangove.
Svaki izbor testne grupe dijeli rangove u dvije skupine, rangove koji pripadaju
testnoj grupi i one koji pripadaju kontrolnoj grupi. Sve kombinacije prikazane su
u sljede\'{c}oj tablici.

\renewcommand{\arraystretch}{1.2}
\begin{equation}
\begin{tabular}{r | c | c | c | c | c}
Testna grupa & (3,4,5) & (2,4,5) & (1,4,5) & (2,3,5) & (1,3,5) \\
Kontrolna grupa & (1,2) & (1,3) & (2,3) & (1,4) & (2,4) \\
\hline
Testna grupa & (2,3,4) & (1,3,4) & (1,2,4) & (1,2,3) & (1,2,5) \\
Kontrolna grupa & (1,5) & (2,5) & (3,5) & (4,5) & (3,4) 
\end{tabular}
\end{equation}
\renewcommand{\arraystretch}{1}

Kao \v{s}to se vidi i iz (1.1.1) pacijenti, a time i njihovi rangovi, mogu se podijeliti
na deset razli\v{c}itih na\v{c}ina. Pod pretpostavkom da smo 3 pacijenta za testnu grupu
izabrali na slu\v{c}ajan na\v{c}in mislimo da \'{c}e bilo koji od ovih 10 na\v{c}ina biti
jednako vjerojatan, tj. imati \'{c}e vjerojatnost $\frac{1}{10}$. Budu\'{c}i da je svaki od
na\v{c}ina jednako vjerojatan  veliki rangovi kod testne grupe ukazivati \'{c}e na uspje\v{s}nost
novog lijeka (vi\v{s}e o ovoj tvrdnji biti \'{c}e prilo\v{z}eno u sljede\'{c}em odlomku).
}

Razmatranja iz Primjera 1.1 lako se generaliziraju. Pretpostavimo da nam je dano $N$ pacijenata
za testiranje i da je na slu\v{c}ajan na\v{c}in izabrano $n$ od po\v{c}etnih $N$, koji \'{c}e
primiti tretman dok \'{c}e ostalih $m=N-n$ slu\v{z}iti kao kontrolna grupa. Iz elementarne teorije
prebrojavanja slijedi da je broj izbora jednak $\binom{N}{n}$. Po
pretpostavci, $n$ pacijenata koji \'{c}e primiti novi tretman izabrano je na slu\v{c}ajan na\v{c}in
(misle\'{c}i ovdje, a i ubudu\'{c}e, da su svi izbori jednako vjerojatni) pa svaki od izbora
ima vjerojatnost $1/\binom{N}{n}$. Na kraju testiranja pacijenti su rangirani (po mogu\'{c}nosti
od nepristranog promatra\v{c}a) s obzirom na ciljani rezultat. Kao i prije, uz ispravnost nulte
hipoteze, na rangiranje ne utje\v{c}e koji su pacijenti u testnoj grupi. Rang svakog od pacijenata
mo\v{z}e se promatrati kao odre\dj{}en (iako nama nepoznat) prije nego li je izvr\v{s}eno
particioniranje na testnu i kontrolnu grupu. Stoga, uz ispravnost nulte hipoteze, svako
pridru\v{z}ivanje rangova jednako je vjerojatno sa vjerojatno\v{s}\'{c}u $1/\binom{N}{n}$.

Zapi\v{s}imo gornju tvrdnju malo formalnije. Neka su $S_{1},\dots,S_{n}$ rangovi testne grupe.
Tada je
\begin{equation}
P_{H_{0}} \left( S_{1}=s_{1}, \dots, S_{n}=s_{n} \right) = \frac{1}{\binom{N}{n}}
\end{equation}

\noindent
gdje je $\{s_{1},\dots,s_{n}\}$ $n$-\v{c}lani podskup od $\{ 1,2,\dots,N \}$,
a $P_{H_{0}}$ ozna\v{c}ava vjerojatnost uz istinitost nulte hipoteze.

Tijekom cijelog odlomka pretpostavljali smo da ne izabiremo pacijente ve\'{c} da su nam dani te
da smo na slu\v{c}ajan na\v{c}in odabrali testnu grupu. Ovakav \'{c}emo model, u kojem
slu\v{c}ajnost ulazi samo kroz odabir testne grupe, zvati \bi{randomizacijski model}
(eng. \textit{randomization model}). Model u kojem se $N$ pacijenata
izabire na odre\dj{}eni na\v{c}in iz neke populacije zovemo \bi{populacijski model}.
U ovom modelu slu\v{c}ajnost ulazi i kroz izbor pacijenata.

Iako nas uglavnom zanima prijenos rezultata na populaciju postoje primjeri
gdje je randomizacijski model dovoljan. Npr. farmer koji \v{z}eli testirati
donosi li novo gnojivo napredak na njegovih nekoliko polja 
i ne zanima ga donosi li napredak na ostalim poljima.

