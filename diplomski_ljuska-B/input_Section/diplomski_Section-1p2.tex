%%%%%%%%%%%%%%%%%%%%%%%%%%%%%%%%%%%%%%%%%%%
%
\section{Randomizacijski model} \label{sec:1p2}
%
\setcounter{equation}{0}

U Primjeru 1.1 re\v{c}eno je da \'{c}e dovoljno veliki rangovi ukazivati na uspje\v{s}nost
novog lijeka. Me\dj{}utim, kada su rangovi $(S_{1},\dots,S_{n})$ dovoljno veliki? Takvi
se zaklju\v{c}ci uglavnom donose pomo\'{c}u neke testne statistike \v{c}ije velike vrijednosti
odgovaraju velikim rangovima. Jedna od takvih test statistika je i
\begin{equation}
W_{S}=S_{1}+\cdots+S_{n} \mathrm{.}
\end{equation}


Statistiku $W_{S}$ definiranu relacijom (1.2.1) zovemo \bi{Wilcoxonova
statistika sume rangova} (eng. \textit{Wilcoxon rank-sum statistic}),
a test definiran s $W_{S}$ \bi{Wilcoxonov test sume rangova} 
(eng. \textit{Wilcoxon rank-sum test}).
Ime su dobili po poznatom ameri\v{c}kom statisti\v{c}aru Franku Wilcoxonu 
koji ih 1945. g. prvi puta upotrijebio u svom radu \cite{Wilcoxon}.

Time smo dobili na\v{c}in za provjeru jesu li nam rangovi dovoljno veliki.
Da bismo izra\v{c}unali kriti\v{c}no podru\v{c}je,
ili $p$-vrijednost, potrebno je poznavati, uz $H_{0}$, distribuciju od $W_{S}$.
Za Primjer 1.1 gdje je $N=5,\ n=3$ distribucija se mo\v{z}e dobiti iz tablice (1.1.1).
Svakom od izbora testne grupe odgovara jedna vrijednost $w$ testne statistike $W_{S}$ 
kao \v{s}to je prikazano u tablici (1.2.2).

\renewcommand{\arraystretch}{1.5}
\begin{equation}
\begin{tabular}{c | c | c | c | c | c | c | c | c | c}
3,4,5 & 2,4,5 & 1,3,5 & 2,3,5  & 1,3,5  & 2,3,4 & 1,3,4 & 1,2,4 & 1,2,3 & 1,2,5 \\
\hline
12 & 11 & 10 & 10 & 9 & 9 & 8 & 7 & 6 & 8
\end{tabular}
\end{equation}
\renewcommand{\arraystretch}{1}

\noindent
Budu\'{c}i da je vjerojatnost svakog izbora jednaka $\frac{1}{10}$ vrijedi

\renewcommand{\arraystretch}{1.5}
\begin{equation}
\begin{tabular}{c | c | c | c | c | c | c | c}
$w$ & 6 & 7 & 8 & 9 & 10 & 11 & 12 \\
\hline
$P_{H_{0}}(W_{S}=w)$ & 0.1 & 0.1 & 0.2 & 0.2 & 0.2 & 0.1 & 0.1
\end{tabular}
\end{equation}
\renewcommand{\arraystretch}{1}

Ove vjerojatnosti tvore distribuciju od $W_{S}$ uz hipotezu $H_{0}$. Distribuciju
uz uvjet istinitosti od $H_{0}$ nazivamo \textit{nulta distribucija}. Iz (1.2.3) slijedi
da za Primjer 1.1. vrijedi
$$P_{H_{0}} (W_{S} \ge 12) = P_{H_{0}} (W_{S} = 12) = 0.1 \mathrm{.}$$
Za razinu zna\v{c}ajnosti $\alpha = 0.1$ nultu hipotezu odbacujemo samo kada je $W_{S}=12$,
tj. kada su rangovi testne grupe 3, 4 i 5.

