%%%%%%%%%%%%%%%%%%%%%%%%%%%%%%%%%%%%%%%%%%%
%
\section{Randomizacijski model za sparene uzorke} \label{sec:2p1}
%
\setcounter{equation}{0}

U prvom smo poglavlju uspore\dj{}ivali dva tretmana kada je $N$ subjekata
dano za testiranje i podijeljeno na slu\v{c}ajan na\v{c}in na testnu i
kontrolnu grupu (randomizacijski model) ili kada se $N$ subjekata
odabire metodom jednostavnog slu\v{c}ajnog uzorkovanja iz neke
populacije (populacijski model). Kod ovakvih modela problem nastaje
kada se subjekti zna\v{c}ajno razlikuju (drasti\v{c}ne razlike
u stupnju bolesti npr.) jer tada razlika mo\v{z}e umanjiti ili
poni\v{s}titi u\v{c}inak tretmana. U takvim slu\v{c}ajevima
u\v{c}inkovitost usporedbe mo\v{z}e se pove\'{c}ati razlaganjem
subjekata u homogenije grupe.

\v{C}e\v{s}to \'{c}e biti lak\v{s}e prona\'{c}i male homogene
grupe, nego velike i uglavnom se subjekti dijele na homogene grupe
po dvoje (parove). Usporedbe s homogenim grupama od dvoje subjekata
nazivamo \bi{sparene usporedbe} (eng. \textit{paired comparisons}).
Primjer je prou\v{c}avanje blizanaca
gdje se svaka grupa sastoji od dva blizanca. Dijeljenje na
parove nije ograni\v{c}eno samo na situacije gdje imamo
prirodno sparivanje, ve\'{c} se mo\v{z}e posti\'{c}i i 
detaljnim prou\v{c}avanjem subjekata. Primjer su pacijenti
koji su u istom stadiju bolesti ili populacije koje imaju
istu geografsku lokaciju.

