\chapter{CGT za jo\v{s} neke rang statistike}

Ovdje \'{c}emo pokazati asimptotsku normalnost statistika
\begin{equation}
S_{N} = \sum_{j=1}^{N} z_{Nj} a_{N} (R_{Nj})
\end{equation}
gdje su $z_{N1}$,\dots,$z_{NN}$ i $a_{N} (1)$,\dots,$a_{N} (N)$ konstante
a $R_{N1}$,\dots,$R_{NN}$ permutacije brojeva $1$,\dots,$N$ sve jednako
vjerojatne (sa vjerojatnosti $1/N!$). Zbog jednostavnosti \v{c}esto \'{c}emo
izostavljati indeks $N$ kod $z$, $a$ i $R$ te pisati
\begin{equation}
S_{N} = \sum_{j=1}^{N} z_{j} a (R_{j}) \ \mathrm{.}
\end{equation}

Va\v{z}an primjer dobijemo za $z_{j}=j$ i $a(j)=1$ kada je $1 \leq j \leq n$ te $a(j)=0$
ina\v{c}e. Tada statistika $S_{N}$ jest Wilcoxonova statistika $W_{S}$
uz nultu hipotezu.

Primijetimo  da se distribucija od $S_{N}$ ne\'{c}e promijeniti ako indekse zamijenimo.
Stoga, bez smanjenja op\'{c}enitosti, mo\v{z}emo pretpostaviti da su konstante $a(j)$
(ili $z_{j}$ ili oboje) u rastu\'{c}em poretku. Distribucija od $S_{N}$ ne\'{c}e se
promijeniti ni ako zamijenimo $a(j)$ i $z_{j}$ jer mo\v{z}emo pisati
$S_{N} = \sum_{j=1}^{N} a(j) z_{{R'}_{j}}$ gdje je ${R'}_{j}$ inverzna permutacija
od $R_{j}$.

